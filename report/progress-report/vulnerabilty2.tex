\section{Vulnerability: Sending more money than balance}
\label{sec:background}
\textbf{Description:} Upon further penetration testing it was discovered that it was possible 
to send money from the account logged in on to another of a larger amount than the account's balance. This may, upon first glance appear to be a bug rather than
necessarily a vulnerability. However, from the previous section we have displayed proof of concept that it is possible for a hacker to gain access to someone's
account without the relevant information. As such he could feasibly send any amount (even ludicrous sums like \£1 billion) from this bank account at Wondough bank
to his bank account with another bank. His bank likely does not have this issue and he has just granted himself huge amounts of money illegally. \\ \\
\textbf{Testing the Vulnerability:} In order to test this vulnerability I wanted to test to see different things the flaw could enable. It is possible to send
transactions to yourself however I do not see how useful this is as a vulnerability to an attacker. However I then decided to add a new user to the database in order
to carry out further tests. \\ \\
\textbf{Mitigation:} The process of creating a transaction occurs in the \verb|DbConnection| file in the \verb|createTransaction()| procedure. This feature already
includes a check to ensure the amount in the transaction is not negative, as follows:
\begin{minted}{java}
if(amount < 0) {
        return false;
}
\end{minted}
And as such in a similar vein an additonal check was then added which makes use of the \verb|getTransactions(user)| and \verb|getAccountBalance()| functions to retrieve
the user's balance and compare it to the amount. As follows:
\begin{minted}{java}
if(amount > getTransactions(user).getAccountBalance()){
    return false;
}
\end{minted}
This was then tested manually with inputs larger than the balance, which were all not entered as transactions, followed by the balance itself and values lower than it, all of
which the fix permitted. However I then set about automating this testing. The automatic testing seen in \verb|test2.java| creates a new user \textit{testUser}, it then tries
to remove his account balance + 1, which should fail if the fix is correct. It then tries the same with his exact account balance, if this works then the fix is perfect. Since
both these conditions are always fine, this vulnerability has been correctly fixed.