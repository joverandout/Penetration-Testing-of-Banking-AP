\section{Vulnerability: Multiple users with duplicate usernames}
\label{sec:background}
\textbf{Description:} The fourth vulnerability I discovered was the ability for multiple users to have the same username. This is a fundamental issue. The username of an account
needs to be its unique identifier that allows a program to identify an individual specific user. If we allow multiple users to share a username then what will happen is that
upon one of those users signing in, the program will return the id of the first user in the database with that username. This means that either if they have by chance chosen a
bad and matching password that access to another user's finances could be granted to another. Otherwise it will be impossible for a user to sign in and to access their finances,
which is without doubt a serious issue.\\ \\
\textbf{Testing the Vulnerability:} To test this vulnerability I first uncommented the section of code that creates a new user. This then created the user \textit{test4User}
twice, with passwords \textit{password} and \textit{password2} respectively. I then signed in to the website, as \textit{test4User@wondoughbank.com}, which I was only able to
do using the password \textit{password}, since upon getting the user the program was returning the first instance in the database. Therefore in a real world scenario the second
user with password \textit{password2} would be entirely unable to access the website indefinitely.\\ \\
\textbf{Mitigation:} Solving this vulnerability occurred in the \verb|createUser()| function within \verb|DbConnection|. A check needed to be applied here to ensure that the
username being used to create the new user did not already exist in the database. To do this the \verb|createUser()| was modified such that it was no longer a \verb|void| but rather
returned a boolean value. True would be if the user had been created and otherwise false. Therefore at the beginning of create user we call \verb|getUser| on the username and
if it doesn't return null then that username already exists in the database and as such the function should return false and go no further.
\begin{minted}{java}
if(getUser(user.getUsername()) != null){
      System.out.println("User already exists");
      return false;
}
\end{minted}
Although I manually tested this in order to ensure its rigorousness an automated test was devised. This test performed 2 actions that both needed to be completed in order for
the test to pass. Firstly a new user needed to be created:
\begin{minted}{java}
WondoughUser testuser1 = new WondoughUser(1,
      "test4User@wondoughbank.com");
testuser1.setSalt(securityConfiguration.generateSalt());
testuser1.setHashedPassword(securityConfiguration.pbkdf2("password",
      testuser1.getSalt()));
testuser1.setIterations(securityConfiguration.getIterations());
testuser1.setKeySize(securityConfiguration.getKeySize());

if((connection.createUser(testuser1)) == false) return "FAILED";
\end{minted}
As can be seen above if the user was not created then the test would fail. After this the test then attempted to create a second user with the same name directly after. The code
for this was identical since we wanted the users to be the same. However, this time if the user was created the test failed:
\begin{minted}{java}
   if((connection.createUser(testuser2)) == true) return "FAILED";
\end{minted}
Then if all that did not happen the test passed. The reason for having the condition of making the first user be created in order to pass was to ensure that the database
connection was working since without this condition the test would pass if it just couldn't connect. Finally I performed some clearing up of the database such that the test data
was removed since it did not represent a real system user and it also needed to be removed so the test would work again next time:
\begin{minted}{java}
String query ="DELETE FROM users"
               +"WHERE username='test4User@wondoughbank.com'";
Statement stmt = connectionToDelete.createStatement();
stmt.executeUpdate(query);
return "PASSED";
\end{minted}
This test passed every time and as such this vulnerability can be considered thoroughly mitigated.
