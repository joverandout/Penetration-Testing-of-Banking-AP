\section{Vulnerability: SQL Injection Attack}
\label{sec:background}
\textbf{Description:} The user data of the bank members is stored in a database, and as such upon clicking the login button the bank performs a simple SQL lookup query in order to
determine if the username exists in the database. However this does not make use of \textit{prepared statements} but rather constructs the SQL query using string concatenation.
This allows the user to use the common technique of including an \textit{OR} followed by a comment to finish the line to gain access. This is demonstrated below:
\begin{minted}{text}
   Username inputted: John
\end{minted}
\begin{minted}{sql}
   SQL: SELECT * FROM users WHERE username='John' LIMIT 1;
\end{minted}
\begin{minted}{text}
   Username inputted: blank' OR 1=1;--
\end{minted}
\begin{minted}{sql}
   SQL: SELECT * FROM users WHERE username='blank' OR 1=1;--' LIMIT 1;
\end{minted}
As can be seen in the query generated in the second instance, an entirely new SQL query has been created with the \textbf{- -} used to comment out the rest of the line.
Since 1=1 will always be true, whether the username exists in the database is irrelevant since the statement will return the first user in the database regardless. This is a dangerous
prospect as may allow a hacker access without relevant information and as such needs to be addressed.\\ \\
\textbf{Testing the Vulnerability:} The Vulnerability gives access for all of the following usernames entered:
 
\begin{minted}{text}
   blank' OR 1=1;--
   blank' OR 'x'='x';--
   blank' OR '*;--
   blank' OR TRUE;--
\end{minted}
Where 1 can be replaced with any number as well as 'x' and 'blank' any string. This can also be used to get any value from any table and since tokens are not time dependent, access
to these would allow the hacker to gain full access to an account. This therefore is a serious security flaw.\\ \\
\textbf{Mitigation:} The issue caused by this flaw is purely as a result of using string concatenation to generate the query. The use of sql prepared statements can be implemented in order
to deny such an attack. A prepared statement works by compiling the statement template before later binding values for the parameters of the statement template. As such the statement knows
when the query ends and  the use of ' to signify the end of the username or - - to turn the remainder of the statement to a comment both no longer work. As such the attack no longer works. A
prepared statement works as follows:\\
 
Rather than creating and executing the query as follows:
\begin{minted}{java}
   PreparedStatement stmt = null;
   String query = "SELECT * FROM users WHERE username='"
       + username + "' LIMIT 1;";
   stmt = this.connection.Statement();
   ResultSet rs = stmt.executeQuery(query);
\end{minted}
It is done as follows:
\begin{minted}{java}
   PreparedStatement stmt = null;
   String query = "SELECT * FROM users WHERE username=? LIMIT 1;";
   stmt = this.connection.prepareStatement(query);
   stmt.setString(1, username);
   ResultSet rs = stmt.executeQuery();
\end{minted}
This is the change I made to the \verb|DbConnection.java| file in order to mitigate this attack. Below is the automated test I designed in order to test the changes appropriately extinguished the
vulnerability.